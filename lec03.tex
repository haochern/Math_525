\documentclass[a4paper]{article}

\usepackage[english]{babel}
\usepackage[utf8x]{inputenc}
\usepackage{amsmath}
\usepackage{amssymb}
\usepackage{graphicx}
\usepackage{bbm}
\usepackage{enumitem}
\usepackage[colorinlistoftodos]{todonotes}
\usepackage[legalpaper, margin=1in]{geometry}
\setlength{\parindent}{0pt}

% Formatting commands
\newcommand{\n}{\hfill\break}
\newcommand{\lemma}[1]{\par\noindent\settowidth{\hangindent}{\textbf{Lemma: }}\textbf{Lemma: }#1\n}
\newcommand{\defn}[1]{\par\noindent\settowidth{\hangindent}{\textbf{Defn: }}\textbf{Defn: }#1\n}
\newcommand{\eg}[1]{\par\noindent\settowidth{\hangindent}{\textbf{E.g.: }}\textbf{E.g.: }#1\n}
\newcommand{\thm}[1]{\par\noindent\settowidth{\hangindent}{\textbf{Thm: }}\textbf{Thm: }#1\n}
\newcommand{\cor}[1]{\par\noindent\settowidth{\hangindent}{\textbf{Cor: }}\textbf{Cor: }#1\n}
\newcommand{\pf}[1]{\par\noindent\settowidth{\hangindent}{\textbf{Proof: }}\textbf{Proof: }#1\n}
\newcommand{\hw}[1]{\par\noindent\settowidth{\hangindent}{\textbf{HW: }}\textbf{HW: }#1\n}
\newcommand{\proven}{\;$\square$\n}
\newcommand{\problem}[1]{\par\noindent{#1}\n}
\newcommand{\problempart}[2]{\par\settowidth{\hangindent}{\textbf{(#1)} \indent{}}\textbf{(#1)} #2\n}
\newcommand{\ptxt}[1]{\textrm{\textnormal{#1}}}
\newcommand{\inlineeq}[1]{\n\centerline{$\displaystyle #1$}}
\newcommand{\pageline}{\noindent\rule{\textwidth}{0.1pt}\n}

% Math
\newcommand{\reals}{\mathbb{R}}
\newcommand{\R}{\reals}
\newcommand{\integers}{\mathbb{Z}}
\newcommand{\Z}{\integers}
\newcommand{\rationals}{\mathbb{Q}}
\newcommand{\Q}{\rationals}
\newcommand{\dom}{\textbf{dom}\;}
\newcommand{\epi}{\textbf{epi}\;}
\newcommand{\prob}{\textbf{prob}}
\newcommand{\ceil}{\text{ceil}}

% Probability
\newcommand{\F}{\mathcal F} 
\newcommand{\parSymbol}{\P}
\newcommand{\Prob}{\mathbb{P}}
\renewcommand{\P}{\Prob}
\newcommand{\Avg}{\mathbb{E}}
\newcommand{\E}{\Avg}
\DeclareMathOperator{\Var}{Var}
\DeclareMathOperator{\cov}{cov}
\DeclareMathOperator{\Unif}{Unif}
\DeclareMathOperator{\Binom}{Binom}
\newcommand{\CI}{\mathrel{\text{\scalebox{1.07}{$\perp\mkern-10mu\perp$}}}}


\title{Lecture 03}
\author{Professor Virginia R. Young\\ \small{Transcribed by Hao Chen}}
\date{September 2, 2022}

\begin{document}

\maketitle

\subsection*{1.4 Conditional prob}

\defn{ Let $(\Omega, \F, \P)$ to be a probability place. Let $B\in\F$ be a fixed set such that $\P(B)>0$. Define the conditional probability given $B$ by \[\P(A\mid B)=\frac{\P(A\cap B)}{\P(B)}\]for any $A\in\F$, where $\P$ is defined by $\P(\Omega)=1$ and $\P(\sqcup^\infty_{i=1}A_i)=\sum^\infty_{i=1}\P(A_i)$.
}

\hw{Show that $\P(\cdot\mid B)$ defines a probability on $(\Omega, \F)$.  ($\cdot$ means any element in $\F$)}

\defn{\textbf{Law of total probability:} Let $A_1, A_2, \dots, A_n$ be a partition of $\Omega\in\F$, $A_i$ are pairwise disjoint and $\cup^n_{i=1}A_i=\Omega$. Assume $\P(A)>0$, \[A_1\cup A_2\cup\dots\cup A_n=\Omega.\] Let $B\in\F$, 
\begin{align*}
    \P(B)&=\P(B\cap\Omega) \\
    &=\P(B\cap(A_1\cup A_2\cup\dots\cup A_n)) \\
    &=\P((B\cap A_1)\cup(B\cap A_2)\cup\dots\cup(B\cap A_n)) \\
\end{align*}
which $B\cap A_i$ are pointwise disjoint for $\forall i$, or 
\[(B\cap A_i)\cap(B\cap A_j)=B\cap(A_i\cap A_j)=\varnothing\]
for $i\neq j$. Thus, 
\begin{align*}
    \P(B)&=\sum^n_{i=1}\P(B\cap A_i) \\
    &=\sum^n_{i=1}\P(B\mid A_i)\P(A_i)
\end{align*}
}

%TODO: pic3

\eg{
Bob possesses five coins, 2 of which are double-headed, 1 is double-tailed, and 2 are normal. Bob shuts his eyes, picks a coin at random, and tosses it.
\begin{enumerate}
    \item What is the prob that the lower face of the coin is a head? 
    \begin{align*}
        \P(\text{lower face}=H) &= \P(\text{lower face}=H\mid\text{2-headed})\P(\text{2-headed}) \\
        &+ \P(\text{lower face} = H\mid\text{2-tailed})\P(\text{2-tailed}) \\
        &+ \P(\text{lower face} = H\mid\text{normal})\P(\text{normal}) \\
        &= 1\cdot\frac{2}{3}+0\cdot\frac{1}{5}+\frac{1}{2}\cdot\frac{2}{5} = \frac{3}{5}
    \end{align*}
    
    \item Bob opens his eyes and sees that the coin that showing heads. What is the probability the lower face is a head? 
    \begin{align*}
        \P(\text{lower face}=H\mid\text{upper face}=H) &= \frac{\P(\text{lower face}=H, \text{upper face}=H)}{\P(\text{upper face}=H)} \\
        &= \frac{\P(\text{both faces}=H)}{\P(\text{upper face}=H)} \\
        &= \frac{2/5}{3/5}=\frac{2}{3}
    \end{align*}
    
    \item Bob shuts his eyes again and re-tosses the same coin. What is the prob that the lower face is a head?
    \begin{align*}
        &\P(\text{lower face}=H\mid\text{the coin has an H}) \\ 
        =&\P(\text{lower face}=H, \text{coin is 2-H})\P(\text{coin is 2-H}\mid\text{the coin has an H}) \\
        +&\P(\text{lower face}=H, \text{coin is normal})\P(\text{coin is normal}\mid\text{the coin has an H}) \\
        =&1\cdot\frac{2}{3}+\frac{1}{2}(1-\frac{2}{3}) \\
        =&\frac{2}{3} + \frac{1}{6} \\
        =&\frac{5}{6}
    \end{align*}
    
    \item Bob open his eyes and sees the coin is showing heads. What is the prob that the lower face is a head?
    \begin{align*}
        \P(\text{lower face}=H\mid\text{this coin upper face}=H)&=\frac{\P(\text{this coin is 2-H})}{\P(\text{this coin lower face}=H)} \\
        &=\frac{2/3}{5/6}=\frac{4}{5}
    \end{align*}
\end{enumerate}
}

\[\P(A\mid B)=\sum^n_{i=1}\P(A\mid B\cap C_i)\P(C_i\mid B)\]


\end{document}
