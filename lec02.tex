\documentclass[a4paper]{article}

\usepackage[english]{babel}
\usepackage[utf8x]{inputenc}
\usepackage{amsmath}
\usepackage{amssymb}
\usepackage{graphicx}
\usepackage{bbm}
\usepackage{enumitem}
\usepackage[colorinlistoftodos]{todonotes}
\usepackage[legalpaper, margin=1in]{geometry}
\setlength{\parindent}{0pt}

% Formatting commands
\newcommand{\n}{\hfill\break}
\newcommand{\lemma}[1]{\par\noindent\settowidth{\hangindent}{\textbf{Lemma: }}\textbf{Lemma: }#1\n}
\newcommand{\defn}[1]{\par\noindent\settowidth{\hangindent}{\textbf{Defn: }}\textbf{Defn: }#1\n}
\newcommand{\eg}[1]{\par\noindent\settowidth{\hangindent}{\textbf{E.g.: }}\textbf{E.g.: }#1\n}
\newcommand{\thm}[1]{\par\noindent\settowidth{\hangindent}{\textbf{Thm: }}\textbf{Thm: }#1\n}
\newcommand{\cor}[1]{\par\noindent\settowidth{\hangindent}{\textbf{Cor: }}\textbf{Cor: }#1\n}
\newcommand{\pf}[1]{\par\noindent\settowidth{\hangindent}{\textbf{Proof: }}\textbf{Proof: }#1\n}
\newcommand{\proven}{\;$\square$\n}
\newcommand{\problem}[1]{\par\noindent{#1}\n}
\newcommand{\problempart}[2]{\par\settowidth{\hangindent}{\textbf{(#1)} \indent{}}\textbf{(#1)} #2\n}
\newcommand{\ptxt}[1]{\textrm{\textnormal{#1}}}
\newcommand{\inlineeq}[1]{\n\centerline{$\displaystyle #1$}}
\newcommand{\pageline}{\noindent\rule{\textwidth}{0.1pt}\n}

% Math
\newcommand{\reals}{\mathbb{R}}
\newcommand{\R}{\reals}
\newcommand{\integers}{\mathbb{Z}}
\newcommand{\Z}{\integers}
\newcommand{\rationals}{\mathbb{Q}}
\newcommand{\Q}{\rationals}
\newcommand{\dom}{\textbf{dom}\;}
\newcommand{\epi}{\textbf{epi}\;}
\newcommand{\prob}{\textbf{prob}}
\newcommand{\ceil}{\text{ceil}}

% Probability
\newcommand{\F}{\mathcal F} 
\newcommand{\parSymbol}{\P}
\newcommand{\Prob}{\mathbb{P}}
\renewcommand{\P}{\Prob}
\newcommand{\Avg}{\mathbb{E}}
\newcommand{\E}{\Avg}
\DeclareMathOperator{\Var}{Var}
\DeclareMathOperator{\cov}{cov}
\DeclareMathOperator{\Unif}{Unif}
\DeclareMathOperator{\Binom}{Binom}
\newcommand{\CI}{\mathrel{\text{\scalebox{1.07}{$\perp\mkern-10mu\perp$}}}}


\title{Lecture 02}
\author{Professor Virginia R. Young\\ \small{Transcribed by Hao Chen}}
\date{August 31, 2022}

\begin{document}

\maketitle

$\F$ is a $\sigma$-algebra of subset of $\Omega$ if it is non-empty, closed under countable union and and closed under complements.

Suppose $A\in\F \implies A^c\in\F$ $A\cup A^c=\Omega\in\F\implies\Omega^c=\varnothing\in\F$

\subsection*{1.3\;\;Probability defined on $(\Omega, \F)$}

Two ways to motivate/define probability:
\begin{itemize}
    \item Frequentist: $A\in\F$, then we think of the probability as the long-range proportion of the times that event $A$ occurs. \[\frac{N(A)}{N}\xrightarrow{N\rightarrow\infty}\P(A)\]
    \item Bayesian/subjective: Suppose you were to win \$1 if A occurs, then $\P(A)$ is the amount you would bet for A to happen.
\end{itemize}

\defn{$\P$ is a probability on $(\Omega, \F)$ if $\P$ maps $\F$ to $[0, 1]$ such that
\begin{enumerate}
    \item $\P(\varnothing)=0$, $\P(\Omega)=1$
    \item If $A_1, A_2, \dots\in\F$ are pairwise disjoint, then\[\P\left(\bigcup^\infty_{i=1}A_i\right)=\sum^\infty_{i=1}\P(A_i)\] $(\Omega, \F, \P)$ is called a probability space.
\end{enumerate}
(Properties of probability) For an arbitrary sequence in $\F$, $B_1, B_2, \dots$ (not necessarily disjoint),\[\bigcup^\infty_{i=1}B_i\in\F\]
So, \[\P\left(\bigcup^\infty_{i=1}B_i\right)\in[0,1]\]
\[\leq\sum^\infty_{i=1}\P(B_i)\]
which is not necessarily equal to the sum of probability

\begin{enumerate}
    \item  $\P(A^c)=1-\P(A)$
$A\cap A^c=\varnothing$, so they are disjoint. Thus, 
\[\P(A\cup A^c)=\P(A)+\P(A^c),\]where $\P(\Omega)=1$. 
\[\implies 1=\P(A)+\P(A^c)\]
\[\P(A^c)=1-\P(A).\]
Aside: If we only require $\P(\Omega)=1$, then
\[\P(\Omega)=\P(\Omega^c)=1-\P(\Omega)=1-1=0.\]
Thus, $\P(\Omega)=0$ follows from $\P(\Omega)=1$ and (countable) additivity of $\P$.

    \item If $A\subset B$, then $\P(A)\leq\P(B)$
\[B=A\cup(B-A)=A\cup(B\cap A^c)\] (disjoint union), where $\P(B\cap A^c)\in[0, 1]$
\[\P(B)=\P(A)+\P(B\cap A^c)\geq\P(A)\]

    \item $\P(A\cup B)=\P(A)+\P(B)-\P(A\cap B)(\leq\P(A)+\P(B))$
\[A\cup B=A\cup(B\cap(A\cap B)^c)\]
\[\P(A\cup B)=\P(A)+\P(B\cap(A\cap B)^c)\]
Aside from (b):
\[\P(B)=\P(A)+\P(B\cap A^c)\]
substituting all $A$ with $A\cap B$, we have 
\[\P(B\cap(A\cap B)^c)=\P(B)-\P(A\cap B)\]
\[\P(A\cup B)\overset{Aside}=\P(A)+\P(B)-\P(A\cap B)\]

    \item More generally, if $A_1, A_2, \dots, A_n\in\F$, then an induction proof shows that
\[\P\left(\bigcup^n_{i=1}A_i\right)=\sum^n_{i=1}\P(A_i)-\sum_{i<j}\P(A_i\cap A_j)+\sum_{i<j<k}\P(A_i\cap A_j\cap A_k)-\dots+(-1)^{n+1}\P(A_1\cap A_2\cap\dots\cap A_n).\]
\end{enumerate}
}

\eg{ $A, B\in\F$, $\P(A)=3/4$, $\P(B)=1/3$, Show that $1/12\leq\P(A\cap B)\leq1/3$
\[A\cap B\subset A\text{\;\;and\;\;}A\cap B\subset B\]
\[\therefore \P(A\cap B)\leq\min(\P(A), \P(B))=\frac{1}{3}\]
\[A\cup B\subset\Omega\implies\P(A\cup B)\leq 1\]
\[\implies\P(A)+\P(B)-\P(A\cap B)\leq 1\]
\[\implies\P(A\cap B)\geq\P(A)+\P(B)-1=\frac{1}{12}\]
}
\end{document}
