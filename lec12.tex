\documentclass[a4paper]{article}

\usepackage[english]{babel}
\usepackage[utf8x]{inputenc}
\usepackage{amsmath}
\usepackage{amssymb}
\usepackage{graphicx}
\usepackage{bbm}
\usepackage{enumitem}
\usepackage[colorinlistoftodos]{todonotes}
\usepackage[legalpaper, margin=1in]{geometry}
\setlength{\parindent}{0pt}

% Formatting commands
\newcommand{\n}{\hfill\break}
\newcommand{\lemma}[1]{\par\noindent\settowidth{\hangindent}{\textbf{Lemma: }}\textbf{Lemma: }#1\n}
\newcommand{\defn}[1]{\par\noindent\settowidth{\hangindent}{\textbf{Defn: }}\textbf{Defn: }#1\n}
\newcommand{\recap}[1]{\par\noindent\settowidth{\hangindent}{\textbf{Recap: }}\textbf{Recap: }#1\n}
\newcommand{\eg}[1]{\par\noindent\settowidth{\hangindent}{\textbf{E.g.: }}\textbf{E.g.: }#1\n}
\newcommand{\thm}[1]{\par\noindent\settowidth{\hangindent}{\textbf{Thm: }}\textbf{Thm: }#1\n}
\newcommand{\cor}[1]{\par\noindent\settowidth{\hangindent}{\textbf{Cor: }}\textbf{Cor: }#1\n}
\newcommand{\hw}[1]{\par\noindent\settowidth{\hangindent}{\textbf{HW: }}\textbf{HW: }#1\n}
\newcommand{\pf}[1]{\par\noindent\settowidth{\hangindent}{\textbf{Proof: }}\textbf{Proof: }#1\n}
\newcommand{\proven}{\;$\square$\n}
\newcommand{\problem}[1]{\par\noindent{#1}\n}
\newcommand{\problempart}[2]{\par\settowidth{\hangindent}{\textbf{(#1)} \indent{}}\textbf{(#1)} #2\n}
\newcommand{\ptxt}[1]{\textrm{\textnormal{#1}}}
\newcommand{\inlineeq}[1]{\n\centerline{$\displaystyle #1$}}
\newcommand{\pageline}{\noindent\rule{\textwidth}{0.1pt}\n}

% Math
\newcommand{\reals}{\mathbb{R}}
\newcommand{\R}{\reals}
\newcommand{\integers}{\mathbb{Z}}
\newcommand{\Z}{\integers}
\newcommand{\rationals}{\mathbb{Q}}
\newcommand{\Q}{\rationals}
\newcommand{\dom}{\textbf{dom}\;}
\newcommand{\epi}{\textbf{epi}\;}
\newcommand{\prob}{\textbf{prob}}
\newcommand{\ceil}{\text{ceil}}

% Probability
\newcommand{\F}{\mathcal F} 
\newcommand{\parSymbol}{\P}
\newcommand{\Prob}{\mathbb{P}}
\renewcommand{\P}{\Prob}
\newcommand{\Avg}{\mathbb{E}}
\newcommand{\E}{\Avg}
\DeclareMathOperator{\Var}{Var}
\DeclareMathOperator{\cov}{cov}
\DeclareMathOperator{\Unif}{Unif}
\DeclareMathOperator{\Binom}{Binom}
\newcommand{\CI}{\mathrel{\text{\scalebox{1.07}{$\perp\mkern-10mu\perp$}}}}

% Text
\newcommand{\tand}{\;\text{and}\;}

\title{Lecture 12}
\author{Professor Virginia R. Young\\ \small{Transcribed by Hao Chen}}
\date{September 30, 2022}

\begin{document}

\maketitle

\subsection*{Jointly distributed r.v.s}

\defn{
    $(X, Y)$ is a bivariate real-valued r.v. on $(\Omega, \F, \P)$ if
    \[(X, Y):\Omega\rightarrow\R^2, \qquad \omega\rightarrow(X(\omega), Y(\omega))\]
    such that $\{\omega\in\Omega:X(\omega)\leq x, Y(\omega)\leq y\}$ is in $\F$, for all $x, y\in\R$. 
    \\\\
    Joint cdf of $X$ and $Y$, or cdf of $(X, Y)$
    \[F_{X,Y}(x, y)=\P(X\leq x, Y\leq y)\]
    From $F_{X,Y}$ we can retrieve the cdfs of $X$ and $Y$.
    \\\\
    Marginal cdf of $X$:
    \[F_X(x)=\P(X\leq x)=\P(X\leq x, Y<\infty)=\lim_{y\rightarrow\infty}F_{X,Y}(x, y)\]
    Similarly, the marginal cdf of $Y$:
    \[F_Y(y)=\lim_{x\rightarrow\infty}F_{X,Y}(x, y)\]
    \begin{enumerate}
        \item 
            If (X, Y) is discrete then its pmf
            \begin{align*}
                p(x, y)&=\P(X=x, Y=y) \\
                &=\P(X\leq x, Y\leq y)-\P(X<x, Y\leq y)-\P(X\leq x, Y<y)+\P(X<x, Y<y) \\
                &=F_{X,Y}(x, y) - F_{X,Y}(x^-, y) - F_{X,Y}(x, y^-) + F_{X,Y}(x^-, y^-)
            \end{align*}
            In the discrete case, to get the marginal pmf of $X$
            \[p_X(x)=\P(X=x)=\sum_y\P(X=x, Y=y)=\sum_y p_{X,Y}(x, y)\]
        \item 
            If (X, Y) is continuous, then $\exists\text{pdf}\;f_{X,Y}(x, y)$ such that
            \[F_{X,Y}(x, y)=\int^y_{-\infty} \int^x_{-\infty}f_{X,Y}(u, v)\;dudv\]
            If $f$ is continuous (where it is non-zero), then
            \begin{align*}
                \frac{\partial F_{X,Y}}{\partial x}&=\int^y_{-\infty}f_{X,Y}(x, v)dv \\
                \frac{\partial^2F_{X,Y}}{\partial y\partial x}&=f_{X,Y}(x, y)
            \end{align*}
            In the continuous case, to get the marginal pdf of $X$, say,
            \[F_X(x)=F_{X,Y}(x, \infty)=\int^\infty_{-\infty}\int^x_{-\infty}f_{X,Y}(u, v)\;dudv\]
            Differentiate with respect to $x$
            \[f_X(x)=\int^\infty_{-\infty}f_{X,Y}(x,y)\;dy\]
            which we call $f_X(x)$ as the marginal pdf of $X$.
    \end{enumerate}
}

\eg{
    Bivariate normal has pdf
    \[f_{X,Y}(x, y)=\frac{1}{2\pi\sigma_1\sigma_2\sqrt{1-\rho^2}}\exp{\left\{-\frac{1}{2(1-\rho^2)}\left[\left(\frac{x-\mu_1}{\sigma_1}\right)^2-2\rho\left(\frac{x-\mu_1}{\sigma_1}\right)\left(\frac{y-\mu_2}{\sigma_2}\right)+\left(\frac{y-\mu_2}{\sigma_2}\right)^2\right]\right\}}\]
    for some $\rho\in(-1, 1)$, $x, y\in\R$.
    \\\\
    Let $z=\frac{y-\mu_2}{\sigma_2}$ and $dz=\frac{dy}{\sigma_2}$,
    \begin{align*}
        f_X(x)&=\int^\infty_{-\infty}f_{X,Y}(x,y)\;dy \\
        &=\frac{e^{-\frac{1}{2(1-\rho^2)}\left(\frac{x-\mu_1}{\sigma_1}\right)^2}}{2\pi\sigma_1\sqrt{1-\rho^2}}\int_{-\infty}^\infty e^{-\frac{1}{2(1-\rho^2)} \left[-2\rho\left(\frac{x-\mu_1}{\sigma_1}\right)z+z^2\right]}\;dz \\
        &=\frac{e^{-\frac{1}{2(1-\rho^2)}\left(\frac{x-\mu_1}{\sigma_1}\right)^2}}{2\pi\sigma_1\sqrt{1-\rho^2}}\cdot e^{\frac{\rho^2}{2(1-\rho^2)}\left(\frac{x-\mu_1}{\sigma_1}\right)^2} \int^\infty_{-\infty} e^{-\frac{1}{2(1-\rho^2)}\left[\rho^2\left(\frac{x-\mu_1}{\sigma_1}\right)^2-2\rho\left(\frac{x-\mu_1}{\sigma_1}\right)z+z^2\right]}\;dz \\
        &=\frac{e^{-\frac{1-\rho^2}{2(1-\rho^2)}\left(\frac{x-\mu_1}{\sigma_1}\right)^2}}{\sqrt{2\pi\sigma_1^2}} \int^\infty_{-\infty} \frac{1}{\sqrt{2\pi(1-\rho^2)}}\cdot e^{-\frac{1}{2(1-\rho^2)}\left[z-\rho\left(\frac{x-\mu_1}{\sigma_1}\right)\right]^2}\;dz
    \end{align*}
    Integrand is pdf of $\text{N}(\rho\frac{x-\mu_1}{\sigma_1}, 1-\rho^2)$, Therefore, 
    \[f_X(x)=\frac{1}{\sqrt{2\pi\sigma^2_1}}e^{\-\frac{1}{2}\left(\frac{x-\mu_1}{\sigma_1}\right)^2}, \quad x\in\R\qquad\implies\qquad X\sim\text{N}(\mu_1, \sigma_1^2)\]
    Similarly, $Y\sim\text{N}(\mu_2, \sigma_2^2)$.
}

\subsection*{Independent r.v.s}
\defn{
    $X$ and $Y$ are independent if 
    \[F_{X,Y}(x, y)=F_X(x)F_Y(y)\]
    for all $x, y\in\R$.
    \\\\
    Or equivalently, if $(X, Y)$ is discrete,
    \[p_{X,Y}(x, y)=p_X(x)p_Y(y)\]
    or if (X,Y) is continuous,
    \[f_{X,Y}(x,y)=f_X(x)f_Y(y)\]
}

\eg{
    Bivariate normal $X$ and $Y$ are independent if and only if $\rho=0$.
    \begin{align*}
        f_{X,Y}(x, y)&=\frac{1}{2\pi\sigma_1\sigma_2\sqrt{1-\rho^2}}\exp{\left\{-\frac{1}{2(1-\rho^2)}\left[\left(\frac{x-\mu_1}{\sigma_1}\right)^2-2\rho\left(\frac{x-\mu_1}{\sigma_1}\right)\left(\frac{y-\mu_2}{\sigma_2}\right)+\left(\frac{y-\mu_2}{\sigma_2}\right)^2\right]\right\}} \\
        &=\frac{1}{\sqrt{2\pi\sigma_1}\sqrt{2\pi\sigma_2}}\exp{\left\{-\frac{1}{2}\left[\left(\frac{x-\mu_1}{\sigma_1}\right)^2+\left(\frac{y-\mu_2}{\sigma_2}\right)^2\right]\right\}} \\
        &=f_X(x)\cdot f_Y(y)
    \end{align*}
}

\hw{
    Prove that, if $(X, Y)$ is continuous, then $F_{X,Y}(x, y)=F_X(x)F_Y(y)\iff f_{X,Y}(x, y)=f_X(x)f_Y(y)$.
}

\end{document}
