\documentclass[a4paper]{article}

\usepackage[english]{babel}
\usepackage[utf8x]{inputenc}
\usepackage{amsmath}
\usepackage{amssymb}
\usepackage{graphicx}
\usepackage{bbm}
\usepackage{enumitem}
\usepackage[colorinlistoftodos]{todonotes}
\usepackage[legalpaper, margin=1in]{geometry}
\setlength{\parindent}{0pt}

% Formatting commands
\newcommand{\n}{\hfill\break}
\newcommand{\lemma}[1]{\par\noindent\settowidth{\hangindent}{\textbf{Lemma: }}\textbf{Lemma: }#1\n}
\newcommand{\defn}[1]{\par\noindent\settowidth{\hangindent}{\textbf{Defn: }}\textbf{Defn: }#1\n}
\newcommand{\eg}[1]{\par\noindent\settowidth{\hangindent}{\textbf{E.g.: }}\textbf{E.g.: }#1\n}
\newcommand{\thm}[1]{\par\noindent\settowidth{\hangindent}{\textbf{Thm: }}\textbf{Thm: }#1\n}
\newcommand{\cor}[1]{\par\noindent\settowidth{\hangindent}{\textbf{Cor: }}\textbf{Cor: }#1\n}
\newcommand{\pf}[1]{\par\noindent\settowidth{\hangindent}{\textbf{Proof: }}\textbf{Proof: }#1\n}
\newcommand{\proven}{\;$\square$\n}
\newcommand{\problem}[1]{\par\noindent{#1}\n}
\newcommand{\problempart}[2]{\par\settowidth{\hangindent}{\textbf{(#1)} \indent{}}\textbf{(#1)} #2\n}
\newcommand{\ptxt}[1]{\textrm{\textnormal{#1}}}
\newcommand{\inlineeq}[1]{\n\centerline{$\displaystyle #1$}}
\newcommand{\pageline}{\noindent\rule{\textwidth}{0.1pt}\n}

% Math
\newcommand{\reals}{\mathbb{R}}
\newcommand{\R}{\reals}
\newcommand{\integers}{\mathbb{Z}}
\newcommand{\Z}{\integers}
\newcommand{\rationals}{\mathbb{Q}}
\newcommand{\Q}{\rationals}
\newcommand{\dom}{\textbf{dom}\;}
\newcommand{\epi}{\textbf{epi}\;}
\newcommand{\prob}{\textbf{prob}}
\newcommand{\ceil}{\text{ceil}}

% Probability
\newcommand{\F}{\mathcal F} 
\newcommand{\parSymbol}{\P}
\newcommand{\Prob}{\mathbb{P}}
\renewcommand{\P}{\Prob}
\newcommand{\Avg}{\mathbb{E}}
\newcommand{\E}{\Avg}
\DeclareMathOperator{\Var}{Var}
\DeclareMathOperator{\cov}{cov}
\DeclareMathOperator{\Unif}{Unif}
\DeclareMathOperator{\Binom}{Binom}
\newcommand{\CI}{\mathrel{\text{\scalebox{1.07}{$\perp\mkern-10mu\perp$}}}}


\title{Lecture 04}
\author{Professor Virginia R. Young\\ \small{Transcribed by Hao Chen}}
\date{September 7, 2022}

\begin{document}

\maketitle

\defn{\textbf{The Monty Hall Problem}
\begin{align*}
    \P(\text{win car if switch})&=\sum^3_{i=1}\P(\text{win car if switch}\mid\text{choose Door i})\cdot\P(\text{choose Door i}) \\ 
    &=\frac{1}{3}\cdot3\cdot\P(\text{win car if switch}\mid\text{Door 1}) \\
    &=\P(\text{win car if switch}\mid\text{Door 1}) \\
    &=\frac{1}{3}+\frac{1}{3}=\frac{2}{3}
\end{align*}
Consider choosing Door 1, 

\begin{center}
\begin{tabular}{c c c}
Location of car & Host opens  & Outcome if switch \\
Door 1          & Door 2 or 3 & no car            \\
Door 2          & Door 3      & win car           \\
Door 3          & Door 2      & win car          
\end{tabular}
\end{center}
}

\subsection*{Independent Events}

\defn{ $(\Omega, \F, \P)$ is a probability space two events $A, B\in\F$ are independent if $\P(A\cap B)=\P(A)\P(B)$. \\
If $A, B$ are independent, then \[\P(A\mid B)=\frac{\P(A\cap B)}{\P(B)}=\frac{\P(A)\P(B)}{\P(B)}=\P(A).\]
}

\eg{Let $\Omega=\{1, 2, \dots, p\}$; $p$ is a prime number. $\F=\{0, 1\}^\Omega$, which $A\subset\Omega$, $\P(A)=\frac{|A|}{p}$, $A\in\F$ means the map $\Omega\rightarrow\{0, 1\}$ such that $A(\omega)=1$ if and only if $\omega\in A$. Show that if $A$ and $B$ are independent events, then at least one of $A$ and $B$ is either $\varnothing$ or $\Omega$.\\
Pf: Suppose $A, B\subset\Omega$ are independent. Then, 
\[P(A\cap B)=\P(A)\P(B)\Rightarrow \frac{|A\cap B|}{p}=\frac{|A|}{p}\cdot\frac{|B|}{p}\]
\[p|A\cap B|=|A|\cdot|B|.\]
Because $p$ is prime, it divides $|A|$ or $|B|$. Without loss of generality, $p$ divides $|A|$. Because $|A|\in\{0, 1, \dots, p\}$, either $|A|=0$ or $|A|=p\Rightarrow\varnothing$ or $\Omega$.
}

\defn{$A_1, A_2, \dots, A_n\in\F$ are independent if for every subset $\{A_{1'}, A_{2'},\dots, A_{r'}\}$ s.t. $r\leq n$, we have \[\P(A_{1'}\cap\dots\cap A_{r'})=\P(A_{1'})\dots\P(A_{r'}).\]
}

\eg{Suppose $\Omega=\{abc, acb, cab, cba, bca, bac, aaa, bbb, bbb\}$, $\F=\{0, 1\}^\Omega$, and each of the nine element in $\Omega$ occurs with equal probability. Let $A_k$ be the event that the $k^{th}$ is "$a$", $k=1, 2, 3$. Show that $A_1, A_2, A_3$ are pairwise independent but not independent. \\
Pf:\[A_1=\{abc, acb, aaa\}, A_2=\{cab, bac, aaa\}, A_3=\{cba, bca, aaa\}\]
\[\P(A_k)=\frac{3}{9}=\frac{1}{3}\]
\[A_1\cap A_2=\{aaa\}=A_1\cap A_3=A_2\cap A_2\]
when $i\neq j$,
\[\P(A_k\cap A_j)=\frac{1}{9}=\frac{1}{3}\cdot\frac{1}{3}=\P(A_k)\P(A_j)\]
\[A_1\cap A_2\cap A_3=\{aaa\}\]
\[\P(A_1\cap A_2\cap A_3)=\frac{1}{9}\neq\P(A_1)\P(A_2)\P(A_3)=\frac{1}{27}\]
}

\eg{In a class, there are 4 freshmen boys, 6 freshmen girls, and 6 sophomore boys. How many sophomore girls must be present if gender and class are to be independent if a student is chosen at random? \\
Ans: Let $x=\#$ sophomore girls. Let $F, S, B, G$ be the events of choosing a freshmen, a sophomore, a boy, or a girl, respectively. Then, 
\begin{align*}
    & \P(F\cap B)=\P(F)\P(B) \\
    \Rightarrow& \frac{4}{16+x}=\frac{10}{16+x}\cdot\frac{10}{16+x} \\
    \Rightarrow& 16+x=25 \\
    \Rightarrow& x=9
\end{align*}
Suppose we only know $\frac{FB}{FG}=\frac{2}{3}$ and we know $SB=6$
\[FB+FG=y,\;\;\frac{FB}{FG}+1=\frac{y}{FG} \Rightarrow \frac{5}{3}=\frac{y}{FG}=\frac{2}{3}\cdot\frac{y}{FB},\;\;FG=\frac{3}{2}FB\]
\[\P(F\cap B)=\P(F)\P(B)\]
\[\frac{\frac{2}{5}y}{y+6+x}=\frac{y}{y+6+x}\cdot\frac{\frac{2}{5}y+6}{y+6+x}\]
\[\frac{2}{5}=\frac{\frac{2}{5}y+6}{y+6+x}\]
\[\frac{2}{5}y+\frac{2}{5}(6+x)=\frac{2}{5}y+6\]
\[6+x=\frac{5}{2}\cdot 6\]
\[x=9\]


}

\end{document}
