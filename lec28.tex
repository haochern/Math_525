\documentclass[a4paper]{article}

\usepackage[english]{babel}
\usepackage[utf8x]{inputenc}
\usepackage{amsmath}
\usepackage{amssymb}
\usepackage{graphicx}
\usepackage{bbm}
\usepackage{enumitem}
\usepackage[colorinlistoftodos]{todonotes}
\usepackage[legalpaper, margin=1in]{geometry}
\setlength{\parindent}{0pt}

% Formatting commands
\newcommand{\n}{\hfill\break}
\newcommand{\lemma}[1]{\par\noindent\settowidth{\hangindent}{\textbf{Lemma: }}\textbf{Lemma: }#1\n}
\newcommand{\defn}[1]{\par\noindent\settowidth{\hangindent}{\textbf{Defn: }}\textbf{Defn: }#1\n}
\newcommand{\recap}[1]{\par\noindent\settowidth{\hangindent}{\textbf{Recap: }}\textbf{Recap: }#1\n}
\newcommand{\eg}[1]{\par\noindent\settowidth{\hangindent}{\textbf{E.g.: }}\textbf{E.g.: }#1\n}
\newcommand{\thm}[1]{\par\noindent\settowidth{\hangindent}{\textbf{Thm: }}\textbf{Thm: }#1\n}
\newcommand{\cor}[1]{\par\noindent\settowidth{\hangindent}{\textbf{Cor: }}\textbf{Cor: }#1\n}
\newcommand{\pf}[1]{\par\noindent\settowidth{\hangindent}{\textbf{Proof: }}\textbf{Proof: }#1\n}
\newcommand{\hw}[1]{\par\noindent\settowidth{\hangindent}{\textbf{HW: }}\textbf{HW: }#1\n}
\newcommand{\proven}{\;$\square$\n}
\newcommand{\problem}[1]{\par\noindent{#1}\n}
\newcommand{\problempart}[2]{\par\settowidth{\hangindent}{\textbf{(#1)} \indent{}}\textbf{(#1)} #2\n}
\newcommand{\ptxt}[1]{\textrm{\textnormal{#1}}}
\newcommand{\inlineeq}[1]{\n\centerline{$\displaystyle #1$}}
\newcommand{\pageline}{\noindent\rule{\textwidth}{0.1pt}\n}

% Math
\newcommand{\reals}{\mathbb{R}}
\newcommand{\R}{\reals}
\newcommand{\integers}{\mathbb{Z}}
\newcommand{\Z}{\integers}
\newcommand{\rationals}{\mathbb{Q}}
\newcommand{\Q}{\rationals}
\newcommand{\dom}{\textbf{dom}\;}
\newcommand{\epi}{\textbf{epi}\;}
\newcommand{\prob}{\textbf{prob}}
\newcommand{\ceil}{\text{ceil}}

% Probability
\newcommand{\F}{\mathcal F} 
\newcommand{\parSymbol}{\P}
\newcommand{\Prob}{\mathbb{P}}
\renewcommand{\P}{\Prob}
\newcommand{\Avg}{\mathbb{E}}
\newcommand{\E}{\Avg}
\DeclareMathOperator{\Var}{Var}
\DeclareMathOperator{\Cov}{Cov}
\DeclareMathOperator{\Unif}{Unif}
\DeclareMathOperator{\Bern}{Bern}
\DeclareMathOperator{\Binom}{Binom}
\DeclareMathOperator{\Poiss}{\text{Poisson}}
\DeclareMathOperator{\Gam}{\text{Gamma}}
\newcommand{\CI}{\mathrel{\text{\scalebox{1.07}{$\perp\mkern-10mu\perp$}}}}

% Text
\newcommand{\tand}{\;\text{and}\;}

\title{Lecture 28}
\author{Professor Virginia R. Young\\ \small{Transcribed by Hao Chen}}
\date{November 14, 2022}

\begin{document}

\maketitle

\defn{
    \textbf{Stochastic Process}
    \\\\
    $\{X_t\}_{t\in I}$ where $X_t$ is a random variable, or the function from $\Omega\rightarrow\R$ such that $\{\omega:X_t(\omega)\leq x\}\in\F_t$, $\forall x\in \R$, $I=\{0,1,2,3,\dots\}$, or $I=\{0,1,2,\dots,T\}$, or $I=[0,T]$, or $I=[0,\infty)$.
    \\\\
    Intuition: $\F_t$ represents info we have at time $t$, As time $t$ progresses, we anticipate that we will have more info so that $\F_t\subset\F_s$, $t\leq s$.
    \\\\
    Special class of stochastic processes is the class of Markov processes:
    \[\P(X_t\leq x\mid X_s, \forall s\in[a,b])=\P(X_t\leq x\mid X_b), \qquad a<b<t\]
    \\\\
    Random walk: Suppose $X_0=2$, then 
    \begin{align*}
        \P(X_1=3\mid X_0=2)&=p \\
        \P(X_1=1\mid X_0=2)&=1-p
    \end{align*}
    More generally,
    \[\P(X_{n+1}=j\mid X_n=i)=\left\{\begin{array}{cc}
        p & \text{if\;\;}j=i+1 \\
        1-p & \text{if\;\;}j=i-1
    \end{array}\right.\]
}

\defn{
    Gambler's Ruin Problem
    \\\\
    For each play of the game, the gambler wins \$1 w.p. $p$, and loses \$1 w.p. $1-p$. Let $X_n$ is gambler's fortune at time $n=0,1,2,\dots$. Find the probability $P_i$ that if the gambler starts with \$$i$, then gambler's fortune will reach \$$N$ before reaching \$0, $\forall i=0,1,\dots,N$.
    \\\\
    Boundary condition: $P_0=0$, $P_N=1$.
    \\\\
    For $i=1,2,\dots, N-1$,
    \begin{align*}
        P_i&=\P(\text{reaching $N$ before 0}\mid X_0=i) \\
        &=\E(\P(\text{reaching $N$ before 0}\mid X_0=i, X_1)) \\
        &=\P(\text{reaching $N$ before 0}\mid X_0=i, X_1=i+1)\P(X_1=i+1\mid X_0=i) \\
        &+\P(\text{reaching $N$ before 0}\mid X_0=i, X_1=i-1)\P(X_1=i-1\mid X_0=i) \\
        &=\P(\text{reaching $N$ before 0}\mid X_1=i+1)\cdot p+\P(\text{reaching $N$ before 0}\mid X_1=i-1)\cdot(1-p)
    \end{align*}
    \begin{align*}
        P_i&=p\cdot P_{i+1}+(1-p)\cdot P_{i-1} \\
        (p+(1-p))\cdot P_i&=p\cdot P_{i+1}+(1-p)\cdot P_{i-1} \\
        (1-p)(P_i-P_{i-1})&=p(P_{i+1}-P_i)
    \end{align*}
    \begin{align*}
        \implies P_{i+1}-P_i&=\frac{1-p}{p}(P_i-P_{i-1}) \\
        &=\left(\frac{1-p}{p}\right)^2(P_{i-1}-P_{i-2}) \\
        &\qquad\vdots \\
        &=\left(\frac{1-p}{p}\right)^i(P_1-P_0)
    \end{align*}
    \begin{align*}
        &\implies P_{i+1}-P_i=\left(\frac{1-p}{p}\right)^i P_1 \\
        &\implies\sum^i_{j=0}(P_{j+1}-P_j)=\left(\sum^i_{j=0}\left(\frac{1-p}{p}\right)^j\right) P_1 \\
        &\implies P_{i+1}=\left(\sum^i_{j=0}\left(\frac{1-p}{p}\right)^j\right)\cdot P_1
    \end{align*}
    1). 
    \[p=\frac{1}{2}\implies\frac{1-p}{p}\implies P_{i+1}=(i+1)P_1\]
    \[1=P_N=N\cdot P_1\implies P_1=\frac{1}{N}\implies P_i=\frac{i}{N}\]
    \\\\
    2). 
    \[p\neq\frac{1}{2}\implies\frac{1-p}{p}\neq1\implies P_{i+1}=\frac{1-\left(\frac{1-p}{p}\right)^{i+1}}{1-\frac{1-p}{p}}\cdot P_1\]
    \[1=P_N=\frac{1-\left(\frac{1-p}{p}\right)^N}{1-\frac{1-p}{p}}\cdot P_1\]
    \[\implies P_1=\frac{1-\frac{1-p}{p}}{1-\left(\frac{1-p}{p}\right)^N}\]
    \begin{align*}
        \implies P_i&=\frac{1-\left(\frac{1-p}{p}\right)^i}{1-\frac{1-p}{p}}\cdot\frac{1-\frac{1-p}{p}}{1-\left(\frac{1-p}{p}\right)^N} \\
        &=\frac{1-\left(\frac{1-p}{p}\right)^i}{1-\left(\frac{1-p}{p}\right)^N}, \qquad i=0,1,2,\dots,N
    \end{align*}
    Suppose $N\rightarrow\infty$, 
    \\\\
    1). for any fixed $i$
    \[p=\frac{1}{2}\implies\lim_{N\rightarrow\infty}P_i=\lim_{N\rightarrow\infty}\frac{1}{N}=0\]
    \\\\
    2a).
    \[p<\frac{1}{2},\;\frac{1-p}{p}>1\implies\lim_{N\rightarrow\infty}P_i=0\]
    \\\\
    2b).
    \[p>\frac{1}{2},\;\frac{1-p}{p}<1\implies\lim_{N\rightarrow\infty}P_i=1-\left(\frac{1-p}{p}\right)^i\]
    \begin{align*}
        i\uparrow&\implies\left(\frac{1-p}{p}\right)^i\downarrow \\
        &\implies1-\left(\frac{1-p}{p}\right)^i\uparrow
    \end{align*}
}




\end{document}
