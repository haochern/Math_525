\documentclass[a4paper]{article}

\usepackage[english]{babel}
\usepackage[utf8x]{inputenc}
\usepackage{amsmath}
\usepackage{amssymb}
\usepackage{graphicx}
\usepackage{bbm}
\usepackage{enumitem}
\usepackage[colorinlistoftodos]{todonotes}
\usepackage[legalpaper, margin=1in]{geometry}
\setlength{\parindent}{0pt}

% Formatting commands
\newcommand{\n}{\hfill\break}
\newcommand{\lemma}[1]{\par\noindent\settowidth{\hangindent}{\textbf{Lemma: }}\textbf{Lemma: }#1\n}
\newcommand{\defn}[1]{\par\noindent\settowidth{\hangindent}{\textbf{Defn: }}\textbf{Defn: }#1\n}
\newcommand{\recap}[1]{\par\noindent\settowidth{\hangindent}{\textbf{Recap: }}\textbf{Recap: }#1\n}
\newcommand{\eg}[1]{\par\noindent\settowidth{\hangindent}{\textbf{E.g.: }}\textbf{E.g.: }#1\n}
\newcommand{\thm}[1]{\par\noindent\settowidth{\hangindent}{\textbf{Thm: }}\textbf{Thm: }#1\n}
\newcommand{\cor}[1]{\par\noindent\settowidth{\hangindent}{\textbf{Cor: }}\textbf{Cor: }#1\n}
\newcommand{\pf}[1]{\par\noindent\settowidth{\hangindent}{\textbf{Proof: }}\textbf{Proof: }#1\n}
\newcommand{\proven}{\;$\square$\n}
\newcommand{\problem}[1]{\par\noindent{#1}\n}
\newcommand{\problempart}[2]{\par\settowidth{\hangindent}{\textbf{(#1)} \indent{}}\textbf{(#1)} #2\n}
\newcommand{\ptxt}[1]{\textrm{\textnormal{#1}}}
\newcommand{\inlineeq}[1]{\n\centerline{$\displaystyle #1$}}
\newcommand{\pageline}{\noindent\rule{\textwidth}{0.1pt}\n}

% Math
\newcommand{\reals}{\mathbb{R}}
\newcommand{\R}{\reals}
\newcommand{\integers}{\mathbb{Z}}
\newcommand{\Z}{\integers}
\newcommand{\rationals}{\mathbb{Q}}
\newcommand{\Q}{\rationals}
\newcommand{\dom}{\textbf{dom}\;}
\newcommand{\epi}{\textbf{epi}\;}
\newcommand{\prob}{\textbf{prob}}
\newcommand{\ceil}{\text{ceil}}

% Probability
\newcommand{\F}{\mathcal F} 
\newcommand{\parSymbol}{\P}
\newcommand{\Prob}{\mathbb{P}}
\renewcommand{\P}{\Prob}
\newcommand{\Avg}{\mathbb{E}}
\newcommand{\E}{\Avg}
\DeclareMathOperator{\Var}{Var}
\DeclareMathOperator{\cov}{cov}
\DeclareMathOperator{\Unif}{Unif}
\DeclareMathOperator{\Binom}{Binom}
\newcommand{\CI}{\mathrel{\text{\scalebox{1.07}{$\perp\mkern-10mu\perp$}}}}

% Text
\newcommand{\tand}{\;\text{and}\;}

\title{Lecture 05}
\author{Professor Virginia R. Young\\ \small{Transcribed by Hao Chen}}
\date{September 9, 2022}

\begin{document}

\maketitle

\subsection*{Bayes' Formula}

\recap{From 1.4, we defined conditional probability. If $\P(B)>0$, then
\[\P(A\mid B)=\frac{\P(A\cap B)}{\P(B)}.\]
\[\frac{\P(A\cap B)}{\P(A)}=\frac{\P(B\cap A)}{\P(A)}=\P(B\mid A)\]
\[\implies\P(A\cap B)=\P(B\mid A)\P(A)\]}

\defn{Bayes' formula
\[\P(A\mid B)=\frac{\P(B\mid A)\P(A)}{\P(B)}\]}

%TODO: ins queuing theory

\eg{A random number $N$ of dice are thrown. $A_i$ is the event that $N=i$ and $\P(A_i)=\frac{1}{2^i}$, $i=1, 2, 3, \dots$. Then sum of the dice is $S$. Assume that each die is fair with $\P(\text{die}=j)=\frac{1}{6}$, $j=1, 2, \dots, 6$, independent of $A_i$.
\begin{enumerate}
    \item Calculate $\P(N=2\mid S=4)$
        \begin{align*}
            &=\frac{\P(S=4\mid N=2)\P(N=2)}{\P(S=4)} \\
            &=\frac{\P(S=4\mid N=2)\P(N=2)}{\sum_{i=1}^\infty\P(S=4\mid N=i)\P(N=i)} \\
            &=\frac{\frac{3}{36}\cdot\frac{1}{2^2}}{(\frac{1}{6}\cdot\frac{1}{2}+\frac{3}{36}\cdot\frac{1}{2^2}+\frac{3}{6^3}\cdot\frac{1}{2^3}+\frac{1}{6^4}\cdot\frac{1}{2^4})} \\
            &=\frac{432}{2197}
        \end{align*}
    \item Calculate $\P(S=4\mid N=\text{even})$
        \begin{align*}
            &=\frac{\P(S=4\;\text{and}\;N=\text{even})}{\P(N=\text{even})} \\
            &=\frac{\P(S=4, N=2)+\P(S=4, N=4)}{\P(N=\text{even})} \\
            &=\frac{\frac{3}{36}\cdot\frac{1}{2^4}+\frac{1}{6^4}\cdot\frac{1}{2^4}}{\frac{1}{2^2}+\frac{1}{2^4}+\dots} \\
            &=\frac{433}{6912}
        \end{align*}
    \item Calculate $\P(N=2\mid S=4\tand\text{first die}=1)$
    \begin{align*}
        &=\frac{\P(N=2\tand S=4\tand X_1=1)}{\P(S=4\tand X_1=1)} \\
        &=\frac{\P(X_1=1\tand X_2=3\tand N=2)}{\P(S=4\tand X_1=1)} \\
        &=\frac{\P(X_1=1\tand X_2=3\mid N=2)\P(N=2)}{\P(S=4, X_1=1)} \\
        &=\frac{\P(X_1=1, X_2=3)\P(N=2)}{\P(S=4, X_1=1)} \\
        &=\frac{\P(X_1=1)\P(X_2=3)\P(N=2)}{\P(S=4, X_1=1)} \\
        &=\frac{\P(X_1=1)\P(X_2=3)\P(N=2)}{\sum_{i=1}^\infty\P(S=4, X_1=1\mid N=i)\P(N=i)} \\
        &=\frac{\frac{1}{6}\cdot\frac{1}{6}\cdot\frac{1}{2^2}}{(\frac{1}{6^2}\cdot\frac{1}{2^2}+\frac{2}{6^3}\cdot\frac{1}{2^3}+\frac{1}{6^4}\cdot\frac{1}{2^4})} \\
        &=\frac{144}{181}
    \end{align*}
    \item Calculate $\P_r:=\P(\text{largest \# show by any die is } r)$
        \[P_r=\P(\text{all die}\leq r)-\P(\text{all die}\leq r-1)\]
        \begin{align*}
            r=1:\;\;&\P_1=\P(\text{all die}=1) \\
            &=\sum^\infty_{i=1}\P(X_1=1=\dots=X_i\mid N=i)\P(N=i) \\
            &=\sum^\infty_{i=1}\frac{1}{6^i}\cdot\frac{1}{2^i}
            =\sum^\infty_{i=1}\frac{1}{12^i} 
            =\frac{\frac{1}{12}}{1-\frac{1}{12}} 
            =\frac{1}{11} \\
            r\geq2:\;\;&\P(\text{all die}\leq r)=\sum^\infty_{i=1}\P(\text{all die}\leq r\mid N=i)\P(N=i) \\ %TODO
            &=\sum^\infty_{i=1}(\frac{r}{6})^i\cdot\frac{1}{2^i}=\frac{r}{12-r} \\
            &\implies\P_2=\frac{2}{12-2}-\frac{1}{11}=\frac{6}{55} \\
            &\;\;\;\;\vdots \\
            &\implies\P_6=\frac{2}{7}
        \end{align*}
        
\end{enumerate}
}



\end{document}
