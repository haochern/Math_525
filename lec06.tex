\documentclass[a4paper]{article}

\usepackage[english]{babel}
\usepackage[utf8x]{inputenc}
\usepackage{amsmath}
\usepackage{amssymb}
\usepackage{graphicx}
\usepackage{bbm}
\usepackage{enumitem}
\usepackage[colorinlistoftodos]{todonotes}
\usepackage[legalpaper, margin=1in]{geometry}
\setlength{\parindent}{0pt}

% Formatting commands
\newcommand{\n}{\hfill\break}
\newcommand{\lemma}[1]{\par\noindent\settowidth{\hangindent}{\textbf{Lemma: }}\textbf{Lemma: }#1\n}
\newcommand{\defn}[1]{\par\noindent\settowidth{\hangindent}{\textbf{Defn: }}\textbf{Defn: }#1\n}
\newcommand{\recap}[1]{\par\noindent\settowidth{\hangindent}{\textbf{Recap: }}\textbf{Recap: }#1\n}
\newcommand{\eg}[1]{\par\noindent\settowidth{\hangindent}{\textbf{E.g.: }}\textbf{E.g.: }#1\n}
\newcommand{\thm}[1]{\par\noindent\settowidth{\hangindent}{\textbf{Thm: }}\textbf{Thm: }#1\n}
\newcommand{\cor}[1]{\par\noindent\settowidth{\hangindent}{\textbf{Cor: }}\textbf{Cor: }#1\n}
\newcommand{\pf}[1]{\par\noindent\settowidth{\hangindent}{\textbf{Proof: }}\textbf{Proof: }#1\n}
\newcommand{\proven}{\;$\square$\n}
\newcommand{\problem}[1]{\par\noindent{#1}\n}
\newcommand{\problempart}[2]{\par\settowidth{\hangindent}{\textbf{(#1)} \indent{}}\textbf{(#1)} #2\n}
\newcommand{\ptxt}[1]{\textrm{\textnormal{#1}}}
\newcommand{\inlineeq}[1]{\n\centerline{$\displaystyle #1$}}
\newcommand{\pageline}{\noindent\rule{\textwidth}{0.1pt}\n}

% Math
\newcommand{\reals}{\mathbb{R}}
\newcommand{\R}{\reals}
\newcommand{\integers}{\mathbb{Z}}
\newcommand{\Z}{\integers}
\newcommand{\rationals}{\mathbb{Q}}
\newcommand{\Q}{\rationals}
\newcommand{\dom}{\textbf{dom}\;}
\newcommand{\epi}{\textbf{epi}\;}
\newcommand{\prob}{\textbf{prob}}
\newcommand{\ceil}{\text{ceil}}

% Probability
\newcommand{\F}{\mathcal F} 
\newcommand{\parSymbol}{\P}
\newcommand{\Prob}{\mathbb{P}}
\renewcommand{\P}{\Prob}
\newcommand{\Avg}{\mathbb{E}}
\newcommand{\E}{\Avg}
\DeclareMathOperator{\Var}{Var}
\DeclareMathOperator{\cov}{cov}
\DeclareMathOperator{\Unif}{Unif}
\DeclareMathOperator{\Binom}{Binom}
\newcommand{\CI}{\mathrel{\text{\scalebox{1.07}{$\perp\mkern-10mu\perp$}}}}

% Text
\newcommand{\tand}{\;\text{and}\;}

\title{Lecture 06}
\author{Professor Virginia R. Young\\ \small{Transcribed by Hao Chen}}
\date{September 12, 2022}

\begin{document}

\maketitle

\subsection*{Random Variables}

\defn{Let $(\Omega, \F, \P)$  be a probability space, X is a real-valued random variable(r.v.) if $X:\Omega\rightarrow\R$ such that $\{\omega:X(\omega)\leq x\}\in\F$ for all $x\in\R$. Then Cumulative Distribution Function(CDF) is \[F_X(x)=\P(X\leq x)\overset{\text{means}}=\P(\{\omega:X(\omega)\leq x\}).\] %TODO mean
}

\eg{$X\sim\text{Bernoulli}(p)$ r.v., $\Omega=\{S, F\}$, $\F=\text{power set}=\sigma\text{-algebra generated by }\{S\}$.
\[\P(\{S\})=p\qquad\P(\{F\})=\P(\{S\}^c)=1-p\]
\[X:\{S, F\}\rightarrow\R\]
\[X(S)=1,\qquad X(F)=0\]
\[F_X(x)=\P(X\leq x)=\left\{\begin{array}{lc}0&x<0\\1-p&0\leq x<1\\1&x\geq1\end{array}\right.\]
}


\eg{$X\sim\text{Geometric}(p)$ \\
Suppose we flip a coin until the first head appears. Let $X$ denote the r.v. that counts the \# of flips required to get the first head. Let $p=$the prob that $H$ appears on any flip. 
\[\Omega=\{(T_1, T_2, \dots, T_{n-1}, H):n=1, 2, \dots\}\]
\[\P(T_1, T_2, \dots, T_{n-1}, H)=(1-p)^{n-1}\cdot p,\qquad n=1, 2, \dots\]
\[X:\Omega\rightarrow\R\]
\[X(T_1, T_2, \dots, T_{n-1}, H)=n\]
Probability Mass Function(PMF):
\begin{align*}
    \P(X=n)&=(1-p)^{n-1}\cdot p,\qquad n=1,2,\dots \\
    &\P(X\leq n)-\P(X<n)=F(x)-F(x^-)
\end{align*}
}

\textbf{A CDF $F$ has the following properties:}
\pf{ If $x<y$, then $F(x)\leq F(y)$.
\begin{align*}
    x<y&\implies\{\omega:X(\omega)\leq x\}\subset\{\omega:X(\omega)\leq y\} \\
    &\implies \P(\{\omega:X(\omega)\leq x\})\leq\P(\{\omega:X(\omega)\leq y\}) \\
    &\implies \P(X\leq x)\leq\P(X\leq y) \\
    &\implies F(x)\leq F(y)
\end{align*}
}

\pf{$\lim_{x\rightarrow\infty}F(x)=1$ \\
Define $A_n=\{\omega\in\Omega:X(\omega)\leq n\}$ where $n=1, 2, \dots$. Then,
\[\Omega=A_1\cup(A_2-A_1)\cup(A_3-(A_1\cup A_2))\cup\dots\]
is a pairwise disjoint union of sets in $\F$. Thus, 
\[1=\P(\Omega)=\P(A_1)+\P(A_2-A_1)+\P(A_3-A_2)+\dots\]
Aside: Consider $A_n\subset A_{n+1}$, 
\[A_{n+1}=A_n\cup(A_{n+1}\cup A_n^c)\implies\P(A_{n+1})=\P(A_n)+\P(A_{n+1}-A_n).\]
Then,
\[\P(A_{n+1}-A_n)=\P(A_{n+1}\cap A_n^c)\]
\[=\P(A_{n+1})-\P(A_n)\]
Therefore, because $\P(A_1)+\sum^n_{i=1}(\P(A_{i+1})-\P(A_i))=\P(A_{n+1})$,
\[1=\P(\Omega)=\P(A_1)+(\P(A_2)-\P(A_1))+(\P(A_3)-\P(A_2))+\dots=\lim_{n\rightarrow\infty}\P(A_n)\]
\[\implies\lim_{n\rightarrow\infty}F(n)=1\]
Since $F$ is non-decreasing,
\[\lim_{x\rightarrow\infty}F(x)=1\]
}

\pf{$\lim_{x\rightarrow-\infty}F(x)=0$ \\
Define $B_n=\{\omega\in\Omega:X(\omega)\leq-n\}$ where $n=1, 2,\dots$. Then $B_n^c=\{X>-n\}$.
A similar argument as in part (b) shows
\[\]
\begin{align*}
    &\lim_{n\rightarrow\infty}\P(B_n^c)=\P(\Omega)=1 \\
    &\implies\lim_{n\rightarrow\infty}(1-\P(B_n))=1 \\
    &\implies\lim_{n\rightarrow\infty}\P(B_n)=0 \\
    &\implies\lim_{n\rightarrow\infty}F(-n)=0 \\
    &\overset{\text{$F$ is non-decreasing}}\implies\lim_{n\rightarrow\infty}F(-x)=0 \\
    &\implies\lim_{n\rightarrow-\infty}F(x)=0
\end{align*}
}

\pf{$\P(X>x)=1-F(x)$ \\
\begin{align*}
    \P(X>x)&=\P(\{\omega:X(\omega)>x\}) \\
    &=\P(\{\omega:X(\omega)\leq x\}^c) \\
    &=1-\P(\{\omega:X(\omega)\leq x\}) \\
    &=1-F(x)
\end{align*}
}

\pf{$\P(x<X\leq y)=F(y)-F(x)$, $x\leq y$ \\
\begin{align*}
    &\P(\{\omega:X(x<\omega)\leq y\}) \\
    =&\P(\{\omega:X(\omega)\leq y\}-\{\omega:X(\omega)\leq x\}) \\
    =&\P(X\leq y)-\P(X\leq x) \\
    =&F(y)-F(x)
\end{align*}
}

\pf{$\P(X=x)=F(x)-\lim_{h\rightarrow0^+}F(x-h)$ \\

}

\pf{$F$ is right-continuous, or $F(x)=\lim_{h\rightarrow0^+}F(x+h)$

}

\end{document}

%TODO: why non-decreasing