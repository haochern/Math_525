\documentclass[a4paper]{article}

\usepackage[english]{babel}
\usepackage[utf8x]{inputenc}
\usepackage{amsmath}
\usepackage{amssymb}
\usepackage{graphicx}
\usepackage{bbm}
\usepackage{enumitem}
\usepackage[colorinlistoftodos]{todonotes}
\usepackage[legalpaper, margin=1in]{geometry}
\setlength{\parindent}{0pt}

% Formatting commands
\newcommand{\n}{\hfill\break}
\newcommand{\lemma}[1]{\par\noindent\settowidth{\hangindent}{\textbf{Lemma: }}\textbf{Lemma: }#1\n}
\newcommand{\defn}[1]{\par\noindent\settowidth{\hangindent}{\textbf{Defn: }}\textbf{Defn: }#1\n}
\newcommand{\eg}[1]{\par\noindent\settowidth{\hangindent}{\textbf{E.g.: }}\textbf{E.g.: }#1\n}
\newcommand{\thm}[1]{\par\noindent\settowidth{\hangindent}{\textbf{Thm: }}\textbf{Thm: }#1\n}
\newcommand{\cor}[1]{\par\noindent\settowidth{\hangindent}{\textbf{Cor: }}\textbf{Cor: }#1\n}
\newcommand{\pf}[1]{\par\noindent\settowidth{\hangindent}{\textbf{Proof: }}\textbf{Proof: }#1\n}
\newcommand{\proven}{\;$\square$\n}
\newcommand{\problem}[1]{\par\noindent{#1}\n}
\newcommand{\problempart}[2]{\par\settowidth{\hangindent}{\textbf{(#1)} \indent{}}\textbf{(#1)} #2\n}
\newcommand{\ptxt}[1]{\textrm{\textnormal{#1}}}
\newcommand{\inlineeq}[1]{\n\centerline{$\displaystyle #1$}}
\newcommand{\pageline}{\noindent\rule{\textwidth}{0.1pt}\n}

% Math
\newcommand{\reals}{\mathbb{R}}
\newcommand{\R}{\reals}
\newcommand{\integers}{\mathbb{Z}}
\newcommand{\Z}{\integers}
\newcommand{\rationals}{\mathbb{Q}}
\newcommand{\Q}{\rationals}
\newcommand{\dom}{\textbf{dom}\;}
\newcommand{\epi}{\textbf{epi}\;}
\newcommand{\prob}{\textbf{prob}}
\newcommand{\ceil}{\text{ceil}}

% Probability
\newcommand{\F}{\mathcal F} 
\newcommand{\parSymbol}{\P}
\newcommand{\Prob}{\mathbb{P}}
\renewcommand{\P}{\Prob}
\newcommand{\Avg}{\mathbb{E}}
\newcommand{\E}{\Avg}
\DeclareMathOperator{\Var}{Var}
\DeclareMathOperator{\cov}{cov}
\DeclareMathOperator{\Unif}{Unif}
\DeclareMathOperator{\Binom}{Binom}
\newcommand{\CI}{\mathrel{\text{\scalebox{1.07}{$\perp\mkern-10mu\perp$}}}}


\title{Lecture 01}
\author{Professor Virginia R. Young\\ \small{Transcribed by Hao Chen}}
\date{August 29, 2022}

\begin{document}

\maketitle
\section*{Chapter 1 Intro to Prob Theory}
\subsection*{1.2\;\;Sample spaces and events}

\defn{The set of all possible outcomes of an experiment is known as the sample space. We denote it by $\Omega$. Also called the state space. $\Omega=\{\omega\}$}
\eg{\begin{itemize}
    \item Ex1: Flip a coin once. $\Omega=\{H, T\}$.
    \item Ex2: Roll a single die. $\Omega=\{1, 2, 3, 4, 5, 6\}$
    \item Ex3: Flip two coins. $\Omega=\{(H, H), (H, T), (T, T)\}$ (unordered pair)
    \item Ex5: Measure the lifetime of an individual $\Omega=(0, \infty)$
\end{itemize}
Events of interest
\begin{itemize}
    \item Ex1" The outcome is not a tail $=\{T\}^c=\Omega-\{T\}=\{H\}$
    \item Ex2" $A=$ outcome is even $=\{2, 4, 6\}$
    \item $B=$ the outcome does not exceed 3 $= \{1, 2, 3\} = \{4, 5, 6\}^c$
    \item Ex5" A = the individual lives at least ten years =$(10, \infty)$
    \item B = the individual dies before age 20 = $(0, 20)$
\end{itemize}
}

\defn{A collection $\F$ of subset of $\Omega$ is called a $\sigma$-algebra if it satisfies the following properties:\\
\begin{enumerate}
    \item $\varnothing\in \F$
    \item if $A_1, A_2, \dots\in \F$ then $\cup^\infty_{i=1}A_i\in \F$
    \item if $A\in \F$, then $A^c\in \F$
\end{enumerate}
}

\eg{\begin{itemize}
    \item (i) $\F={\varnothing, \Omega}$ is a $\sigma$-algebra.
    \item (ii) if $A\subset\Omega$, then $\F=\{\varnothing, A, A^c, \Omega\}$ is a $\sigma$-algebra.
    \item (iii) Borel $\sigma$-algebra of $\R$ = $\sigma$-algebra generated by $(a, b)$, $a<b$, $a,b\in\R$.
\end{itemize}

$A+B=A\cup B$\\
$A-B=A\cap B^c$\\
$\Omega-B=\Omega\cap B^c=B^c$\\
}

\pf{

(a) $\varnothing\in\F$ by definition of $\F$.\\
(b) Consider a sequence of up to 4 elements in $\F$,
\begin{itemize}
    \item If $A$ and $A^c$ are in the sequence, then the union is $\omega\in\F$.
    \item If $\Omega$ is the sequence, then the union is $\Omega\in\F$.
    \item Otherwise, the union is either $\varnothing$, $A$, or $A^c\in\F$.
\end{itemize}
(c) $\varnothing^c=\Omega$, $\{A\}^c=\{A^c\}$, $\Omega^c=\varnothing$, $\{A^c\}^c=\{A\}$
Therefore, $\F$ is closed under $^c$.

Let $\F$ be a $\sigma$-algebra, and consider a sequence $A_1, A_2, \dots\in \F$, then, $A_i^c\in\F$, $i=1, 2, \dots$.\\
Let $\omega\in\cup^\infty_{i=1}A_i^c$; thus, $\exists j=1,2,\dots$ s.t. $\omega\in A^c_j\implies\omega\notin A\implies\omega\notin\cap^\infty_{i=1}A_i\implies\omega\in(\cap^\infty_{i=1}A_i)^c$. Prove that $\cup^\infty_{i=1}A^c_i\subset(\cap^\infty_{i=1}A_i)^c$. $\therefore$ $\cap^\infty_{i=1}A_i\in\F$
}

\eg{
$\cap^\infty_{n=1}(a-1/n, b+1/n)=[a, b]$
}

\end{document}
