\documentclass[a4paper]{article}

\usepackage[english]{babel}
\usepackage[utf8x]{inputenc}
\usepackage{amsmath}
\usepackage{amssymb}
\usepackage{graphicx}
\usepackage{bbm}
\usepackage{enumitem}
\usepackage[colorinlistoftodos]{todonotes}
\usepackage[legalpaper, margin=1in]{geometry}
\setlength{\parindent}{0pt}

% Formatting commands
\newcommand{\n}{\hfill\break}
\newcommand{\lemma}[1]{\par\noindent\settowidth{\hangindent}{\textbf{Lemma: }}\textbf{Lemma: }#1\n}
\newcommand{\defn}[1]{\par\noindent\settowidth{\hangindent}{\textbf{Defn: }}\textbf{Defn: }#1\n}
\newcommand{\recap}[1]{\par\noindent\settowidth{\hangindent}{\textbf{Recap: }}\textbf{Recap: }#1\n}
\newcommand{\eg}[1]{\par\noindent\settowidth{\hangindent}{\textbf{E.g.: }}\textbf{E.g.: }#1\n}
\newcommand{\thm}[1]{\par\noindent\settowidth{\hangindent}{\textbf{Thm: }}\textbf{Thm: }#1\n}
\newcommand{\cor}[1]{\par\noindent\settowidth{\hangindent}{\textbf{Cor: }}\textbf{Cor: }#1\n}
\newcommand{\pf}[1]{\par\noindent\settowidth{\hangindent}{\textbf{Proof: }}\textbf{Proof: }#1\n}
\newcommand{\proven}{\;$\square$\n}
\newcommand{\problem}[1]{\par\noindent{#1}\n}
\newcommand{\problempart}[2]{\par\settowidth{\hangindent}{\textbf{(#1)} \indent{}}\textbf{(#1)} #2\n}
\newcommand{\ptxt}[1]{\textrm{\textnormal{#1}}}
\newcommand{\inlineeq}[1]{\n\centerline{$\displaystyle #1$}}
\newcommand{\pageline}{\noindent\rule{\textwidth}{0.1pt}\n}

% Math
\newcommand{\reals}{\mathbb{R}}
\newcommand{\R}{\reals}
\newcommand{\integers}{\mathbb{Z}}
\newcommand{\Z}{\integers}
\newcommand{\rationals}{\mathbb{Q}}
\newcommand{\Q}{\rationals}
\newcommand{\dom}{\textbf{dom}\;}
\newcommand{\epi}{\textbf{epi}\;}
\newcommand{\prob}{\textbf{prob}}
\newcommand{\ceil}{\text{ceil}}

% Probability
\newcommand{\F}{\mathcal F} 
\newcommand{\parSymbol}{\P}
\newcommand{\Prob}{\mathbb{P}}
\renewcommand{\P}{\Prob}
\newcommand{\Avg}{\mathbb{E}}
\newcommand{\E}{\Avg}
\DeclareMathOperator{\Var}{Var}
\DeclareMathOperator{\cov}{cov}
\DeclareMathOperator{\Unif}{Unif}
\DeclareMathOperator{\Binom}{Binom}
\newcommand{\CI}{\mathrel{\text{\scalebox{1.07}{$\perp\mkern-10mu\perp$}}}}

% Text
\newcommand{\tand}{\;\text{and}\;}

\title{Lecture 11}
\author{Professor Virginia R. Young\\ \small{Transcribed by Hao Chen}}
\date{September 28, 2022}

\begin{document}

\maketitle

\subsection*{Expectations}
\recap{\[\E X=\int^\infty_{-\infty} x\;dF(x)\]
}

\eg{$X\sim U(a, b)$
\[\E X=\int^b_a xf(x)\;dx=\frac{b+a}{2}\]
}

\eg{$X\sim \text{Gamma}(\alpha, \lambda)$
\begin{align*}
    \E X &=\int^\infty_0 xf(x)\;dx \\
    &=\int^\infty_0 x\cdot\frac{x^{\alpha-1}e^{-\lambda x}}{\Gamma(\alpha)}\lambda^\alpha\;dx \\
    &=\frac{\lambda^\alpha}{\Gamma(\alpha)}\int^\infty_0 x^{(\alpha+1)-1}e^{-\lambda x}\;dx \\
    &=\frac{\lambda^\alpha}{\Gamma(\alpha)}\cdot\frac{\Gamma(\alpha+1)}{\lambda^{\alpha+1}} \int^\infty_0 \frac{\lambda^{\alpha+1}x^{(\alpha+1)-1}e^{-\lambda x}}{\Gamma(\alpha+1)}\;dx
\end{align*}
where we have
\[\int^\infty_0 \frac{\lambda^{\alpha+1}x^{(\alpha+1)-1}e^{-\lambda x}}{\Gamma(\alpha+1)}\;dx=1\]
Thus, 
\[\E X=\frac{\Gamma(\alpha+1)}{\Gamma(\alpha)}\frac{1}{\lambda}=\frac{\alpha}{\lambda}\]
}

\eg{Special case of the $\text{Gamma}(\alpha, \lambda)$ with $\alpha=1$ is $\text{Exp}(\lambda)$
Another way to compute $\E$. Let $S(x)=1-F(x)=\P(X>x)$, or survival function of $X$.
\begin{align*}
    \E X&=\int_0^\infty(1-F(x))\;dx \\
    &=\int_0^\infty S(x)\;dx \\
    &=\int_0^\infty e^{-\lambda x}\;dx \\
    &=-\frac{1}{\lambda}e^{-\lambda x}\bigg\vert^\infty_0=\frac{1}{\lambda}
\end{align*}
Therefore, $\E X=\frac{1}{\lambda}$.
}

\eg{$X\sim \text{Pareto}(\alpha, \theta)$
\[S(x)=\left\{\begin{array}{lc}1&x<0\\\left(\frac{\theta}{\theta+x}\right)^\alpha&x\geq0\end{array}\right.\]
\[f(x)=F'(x)=-S'(x)\]
\begin{align*}
    \E X&=\int^\infty_0 \left(\frac{\theta}{\theta+x}\right)^\alpha\;dx \\
    &=\theta^\alpha\frac{(\theta+x)^{-\alpha+1}}{-\alpha+1}\bigg\vert^\infty_{x=0}
\end{align*}
\[\E X=\left\{\begin{array}{lc}\infty&\alpha\leq1\\\frac{\theta}{\alpha-1}&\alpha>1\end{array}\right.\]
}

\eg{$X\sim N(\mu, \sigma^2)$
\[f(x)=\frac{1}{\sqrt{2\pi\sigma^2}}e^{-\frac{1}{2}\left(\frac{x-\mu}{\sigma}\right)^2}\]
$f(x)$ symmetric with respect to $x=\mu$ $\implies$ $\E X=\mu$.
\[\E X=\int^\infty_{-\infty} x\cdot\frac{1}{\sqrt{2\pi\sigma^2}}e^{-\frac{1}{2}\left(\frac{x-\mu}{\sigma}\right)^2}\;dx\]
Let $z=\frac{x-\mu}{\sigma}$, then $dz=\frac{dx}{\sigma}$,
\begin{align*}
    \E X&=\int^\infty_{-\infty}(\sigma z+\mu)\frac{1}{\sqrt{2\pi}}e^{-\frac{1}{2}z^2}\;dz \\
    &=\frac{\sigma}{\sqrt{2\pi}}\int^\infty_{-\infty}ze^{-\frac{1}{2}z^2}\;dz+\mu\int^\infty_{-\infty}\frac{1}{\sqrt{2\pi}} e^{-\frac{1}{2}z^2}\;dz \\
    &=\left(-e^{-\frac{1}{2}z^2}\right)\bigg\vert^\infty_{z=-\infty}+\mu \\
    &=\mu
\end{align*}
}

\defn{Expectation of a function g of X
\begin{align*}
    \E(g(X))&=\int^\infty_{-\infty} g(x)\;dF(x) \\
    &=\lim_{\|\pi\|\rightarrow0}\sum^\infty_{i=-\infty}g(x_i^*)(F(x_i)-F(x_{i-1}))
\end{align*}
where $x_{-\infty}=-\infty$ and $x_\infty=\infty$.
\\\\
If $X$ is discrete with pmf $p$,
\[\E(g(X))=\sum_{x_k}g(x_k)p(x_k)\]
If $X$ is continuous with pdf $f$,
\[\E(g(X))=\int^\infty_{-\infty} g(x)f(x)\;dx\]
$\E$ is a linear operator
\[g(x)=ax+b,\qquad a, b\in\R\]
\begin{align*}
    \E(aX+b)&=\int_{-\infty}^\infty(ax+b)\;F(x) \\
    &=a\int_{-\infty}^\infty x\;dF(x)+b\int_{-\infty}^\infty1\;dF(x) \\
    &=a\E X+b
\end{align*}
}

\defn{Variance of $X$: $g(x)=(x-\E X)^2$
\begin{align*}
    \Var X&=\E[(X-\E X)^2] \\
    &=\E[X^2-2\E X\cdot X+(\E X)^2] \\
    &=\E(X^2)-2\E X\cdot\E X+(\E X)^2 \\
    &=\E(X^2)-(\E X)^2
\end{align*}
Typically, we call $\E(X^k)$ as $k^{\underline{th}}$ moment, so $\E(X^2)$ is second moment. \\
Similarly, we call $\E((X-\E X)^k)$ as $k^{\underline{th}}$ central moment, so $\Var X$ is $2^{nd}$ central moment. 
\\\\
For $a, b\in\R$, 
\begin{align*}
    &\Var(aX+b) \\
    =&\E((aX+b-\E(aX+b))^2) \\
    =&\E((aX+b-(a\E X+b))^2) \\
    =&\E((aX-a\E X)^2) \\
    =&\E(a^2(X-\E X)^2) \\
    =&a^2\E((X-\E X^2) \\
    =&a^2\Var X
\end{align*}
}

\eg{$X\sim N(\mu, \sigma^2)$
Given that $\E X=\mu$,
\[\Var(X)=\int^\infty_{-\infty} (x-\mu)^2 \frac{1}{\sqrt{2\pi\sigma^2}}e^{-\frac{1}{2}\left(\frac{x-\mu}{\sigma}\right)^2}\;dx\]
Let $z=\frac{x-\mu}{\sigma}$, then $dz=\frac{dx}{\sigma}$,
\begin{align*}
    \Var X&=\int^\infty_{-\infty}\sigma^2 z^2\frac{1}{\sqrt{2\pi}}e^{-\frac{1}{2}z^2}\;dz \\
    &= \frac{\sigma^2}{\sqrt{2\pi}}\int^\infty_{-\infty}z\left(ze^{-\frac{1}{2}z^2}\;dz\right)
\end{align*}
Consider that $u=z$ and $dt=ze^{-\frac{1}{2}z^2}\;dz$,
\begin{align*}
    \Var X&=\sigma^2\frac{1}{\sqrt{2\pi}}\left[-ze^{-\frac{1}{2}z^2}\bigg\vert^\infty_{-\infty}+\int_{-\infty}^\infty e^{-\frac{1}{2}z^2}\;dz\right] \\
    &=\sigma^2\frac{1}{\sqrt{2\pi}}\int_{-\infty}^\infty e^{-\frac{1}{2}z^2}\;dz=\sigma^2
\end{align*}
}



\end{document}
