\documentclass[a4paper]{article}

\usepackage[english]{babel}
\usepackage[utf8x]{inputenc}
\usepackage{amsmath}
\usepackage{amssymb}
\usepackage{graphicx}
\usepackage{bbm}
\usepackage{enumitem}
\usepackage[colorinlistoftodos]{todonotes}
\usepackage[legalpaper, margin=1in]{geometry}
\setlength{\parindent}{0pt}

% Formatting commands
\newcommand{\n}{\hfill\break}
\newcommand{\lemma}[1]{\par\noindent\settowidth{\hangindent}{\textbf{Lemma: }}\textbf{Lemma: }#1\n}
\newcommand{\defn}[1]{\par\noindent\settowidth{\hangindent}{\textbf{Defn: }}\textbf{Defn: }#1\n}
\newcommand{\recap}[1]{\par\noindent\settowidth{\hangindent}{\textbf{Recap: }}\textbf{Recap: }#1\n}
\newcommand{\eg}[1]{\par\noindent\settowidth{\hangindent}{\textbf{E.g.: }}\textbf{E.g.: }#1\n}
\newcommand{\thm}[1]{\par\noindent\settowidth{\hangindent}{\textbf{Thm: }}\textbf{Thm: }#1\n}
\newcommand{\cor}[1]{\par\noindent\settowidth{\hangindent}{\textbf{Cor: }}\textbf{Cor: }#1\n}
\newcommand{\pf}[1]{\par\noindent\settowidth{\hangindent}{\textbf{Proof: }}\textbf{Proof: }#1\n}
\newcommand{\proven}{\;$\square$\n}
\newcommand{\problem}[1]{\par\noindent{#1}\n}
\newcommand{\problempart}[2]{\par\settowidth{\hangindent}{\textbf{(#1)} \indent{}}\textbf{(#1)} #2\n}
\newcommand{\ptxt}[1]{\textrm{\textnormal{#1}}}
\newcommand{\inlineeq}[1]{\n\centerline{$\displaystyle #1$}}
\newcommand{\pageline}{\noindent\rule{\textwidth}{0.1pt}\n}

% Math
\newcommand{\reals}{\mathbb{R}}
\newcommand{\R}{\reals}
\newcommand{\integers}{\mathbb{Z}}
\newcommand{\Z}{\integers}
\newcommand{\rationals}{\mathbb{Q}}
\newcommand{\Q}{\rationals}
\newcommand{\dom}{\textbf{dom}\;}
\newcommand{\epi}{\textbf{epi}\;}
\newcommand{\prob}{\textbf{prob}}
\newcommand{\ceil}{\text{ceil}}

% Probability
\newcommand{\F}{\mathcal F} 
\newcommand{\parSymbol}{\P}
\newcommand{\Prob}{\mathbb{P}}
\renewcommand{\P}{\Prob}
\newcommand{\Avg}{\mathbb{E}}
\newcommand{\E}{\Avg}
\DeclareMathOperator{\Var}{Var}
\DeclareMathOperator{\Cov}{Cov}
\DeclareMathOperator{\Unif}{Unif}
\DeclareMathOperator{\Binom}{Binom}
\newcommand{\CI}{\mathrel{\text{\scalebox{1.07}{$\perp\mkern-10mu\perp$}}}}

% Text
\newcommand{\tand}{\;\text{and}\;}

\title{Lecture 16}
\author{Professor Virginia R. Young\\ \small{Transcribed by Hao Chen}}
\date{October 10, 2022}

\begin{document}

\maketitle

\begin{align*}
    \Cov(X,Y)&=\E[(X-\E X)(Y=\E Y)] \\
    &=\E(XY)-\E X\cdot \E Y
\end{align*}
Assume $\Var X>0$, $\Var Y>0$
\[\rho_{X,Y}=\frac{\Cov(X,Y)}{\sqrt{\Var X\cdot\Var Y}}\]
$\rho_{X,Y}=\pm1\iff \exists a, b\in\R(a\neq0)$ such that $\P(X=aY+b)=1$

\pf{
    \begin{enumerate}
        \item $\rho_{X,Y}=1\implies\P(X=aY+b)=1$ where $a>0$, $\exists a, b$
        \begin{align*}
            1=\rho_{X,Y}&=\frac{\Cov(X,Y)}{\sqrt{\Var X\cdot\Var Y}} \\
            1&=\frac{\E[(X-\E X)(Y-\E Y)]}{\sqrt{\Var X\cdot\Var Y}} \\
            1&=\E\left[\frac{X-\E X}{\sqrt{\Var X}}\cdot\frac{Y-\E Y}{\sqrt{\Var Y}}\right]
        \end{align*}
        Define $Z=\frac{X-\E X}{\sqrt{\Var X}}$ and $V=\frac{Y-\E Y}{\sqrt{\Var Y}}$
        \[\E Z=0=\E V\]
        \begin{align*}
            \Var Z&=\E(Z^2)=1 \\
            \Var V&=\E(V^2)=1 \\
        \end{align*}
        \[\E(ZV)=1\]
        Consider $\E[(Z-V)^2]\geq0$
        \begin{align*}
            \implies&\E[Z^2-2ZV+V^2] \\
            =&\E(Z^2)-2\E(ZV)+\E(V^2) \\
            =&1-2+1=0
        \end{align*}
        \begin{align*}
            \E[(Z-V)^2]=0&\implies\P[(Z-V)^2=0]=1 \\
            &\implies\P[(Z-V=0]=1 \\
            &\implies\P[Z=V]=1
        \end{align*}
        (Similarly if $\Var X=\E[(X-\E X)^2]=0$, then $\P(X=\E X)=1$)
        \begin{align*}
            1&=\P\left(\frac{X-\E X}{\sqrt{\Var X}}=\frac{Y-\E Y}{\sqrt{\Var Y}}\right) \\
            &=\P\left(X=\E X+\frac{\sqrt{\Var X}}{\sqrt{\Var Y}}(Y-\E Y)\right) \\
            &=\P(X=aY+b), \qquad a=\frac{\sqrt{\Var X}}{\sqrt{\Var Y}}>0
        \end{align*}
        
        \item  $\rho_{X,Y}=-1$ then $\exists a<0, b\in\R$ such that $\P(X=aY+b)=1$. \\\\
        HW:
        \begin{enumerate}
            \item rewrite $\rho_{X,Y}=\E(ZV)=-1$
            \item consider $\E[(Z+V)^2]$
        \end{enumerate}
    \end{enumerate}
    If $\P(X=aY+b)=1$ for some $a\neq0,b\in\R$ then $\rho_{X,Y}=\pm1$.
    \begin{align*}
        \rho_{X,Y}&=\frac{\E[(X-\E X)(Y-\E Y)]}{\sqrt{\Var X\cdot\Var Y}} \\
        &=\frac{\E[(aY+b-\E(aY+b))(Y-\E Y)]}{\sqrt{\Var(aY+b)\cdot\Var Y}} \\
        &=\dots \\
        &=\pm1
    \end{align*}
}

\subsection*{Sum of two independent random variables}

\defn{
    Suppose $X$ and $Y$ are continuous, $Z=X+Y$. Then the joint pdf of $X,Y$ is
    \[f_{X,Y}(x,y)=f_X(x)f_Y(y)\]
    to get the pdf of $Z$, first calculate the cdf of $Z$:
    \begin{align*}
        F_Z(z)&=\P(Z\leq z)=\P(X+Y\leq Z) \\
        &=\underset{x+y\leq z}{\int\int}f_{X,Y}(x,y)\;dxdy \\
        &=\int_{-\infty}^\infty\int_{-\infty}^{z-y} f_{X,Y}(x,y)\;dxdy \\
        &=\int_{-\infty}^\infty F_X(z-y)f_Y(y)\;dy
    \end{align*}
    \[f_Z(z)=\int_{-\infty}^\infty f_X(z-y)f_Y(y)\;dy\]
    If $X,Y\geq0$ rvs, then
     \[f_Z(z)=\int_0^z f_X(z-y)f_Y(y)\;dy\]
}

\eg{
    $X,Y$ are independent $\text{Exp}(\lambda)$ rvs $f_X(x)=\lambda e^{-\lambda x}$, $x\geq0$
    \begin{align*}
        f_Z(z)&=\int^z_0\lambda e^{-\lambda(z-y)}\cdot\lambda e^{-\lambda y}\;dy,\qquad z\geq0 \\
        &=\lambda^2e^{-\lambda z}\int^z_0 1\;dy \\
        &=\lambda^2ze^{-\lambda z} \implies Z\sim\text{Gamma}(2,\lambda)
    \end{align*}
    HW: Show $X+Y\sim\text{Gamma}(\alpha+\beta,\lambda)$
    \[\left.\begin{array}{c}
        X\sim\text{Gamma}(\alpha,\lambda)  \\
        Y\sim\text{Gamma}(\beta,\lambda)
    \end{array}\right\}\text{independent rvs},\;\;\alpha,\beta>0\]
}

\end{document}
